%% Based on a TeXnicCenter-Template by Gyorgy SZEIDL.
%%%%%%%%%%%%%%%%%%%%%%%%%%%%%%%%%%%%%%%%%%%%%%%%%%%%%%%%%%%%%

%------------------------------------------------------------
%
\documentclass[landscape,twoside]{article}%
%Options -- Point size:  10pt (default), 11pt, 12pt
%        -- Paper size:  letterpaper (default), a4paper, a5paper, b5paper
%                        legalpaper, executivepaper
%        -- Orientation  (portrait is the default)
%                        landscape
%        -- Print size:  oneside (default), twoside
%        -- Quality      final(default), draft
%        -- Title page   notitlepage, titlepage(default)
%        -- Columns      onecolumn(default), twocolumn
%        -- Equation numbering (equation numbers on the right is the default)
%                        leqno
%        -- Displayed equations (centered is the default)
%                        fleqn (equations start at the same distance from the right side)
%        -- Open bibliography style (closed is the default)
%                        openbib
% For instance the command
%           \documentclass[a4paper,12pt,leqno]{article}
% ensures that the paper size is a4, the fonts are typeset at the size 12p
% and the equation numbers are on the left side
%
\usepackage[T1]{fontenc}
\usepackage{amsmath}%
\usepackage{amsfonts}%
\usepackage{amssymb}%
\usepackage{graphicx}
\usepackage{multicol}
\usepackage[landscape]{geometry}
\usepackage{algorithm}
\usepackage{algpseudocode}
\usepackage{tabularx}
\usepackage[mono=false]{libertine}
\usepackage{enumitem}
\usepackage{caption}
%-------------------------------------------
\newtheorem{theorem}{Theorem}
\newtheorem{acknowledgement}[theorem]{Acknowledgement}
\newtheorem{axiom}[theorem]{Axiom}
\newtheorem{case}[theorem]{Case}
\newtheorem{claim}[theorem]{Claim}
\newtheorem{conclusion}[theorem]{Conclusion}
\newtheorem{condition}[theorem]{Condition}
\newtheorem{conjecture}[theorem]{Conjecture}
\newtheorem{corollary}[theorem]{Corollary}
\newtheorem{criterion}[theorem]{Criterion}
\newtheorem{definition}[theorem]{Definition}
\newtheorem{example}[theorem]{Example}
\newtheorem{exercise}[theorem]{Exercise}
\newtheorem{lemma}[theorem]{Lemma}
\newtheorem{notation}[theorem]{Notation}
\newtheorem{problem}[theorem]{Problem}
\newtheorem{proposition}[theorem]{Proposition}
\newtheorem{remark}[theorem]{Remark}
\newtheorem{solution}[theorem]{Solution}
\newtheorem{summary}[theorem]{Summary}
\newenvironment{proof}[1][Proof]{\textbf{#1.} }{\ \rule{0.5em}{0.5em}}

\geometry{top=0.5in, left=0.5in, right=0.5in, bottom=0.5in}

\pagestyle{empty}

% Redefine section commands to use less space
\makeatletter
\renewcommand{\section}{\@startsection{section}{1}{0mm}%
                                {-1ex plus -.5ex minus -.2ex}%
                                {0.5ex plus .2ex}%x
                                {\normalfont\large\bfseries}}
\renewcommand{\subsection}{\@startsection{subsection}{2}{0mm}%
                                {-1explus -.5ex minus -.2ex}%
                                {0.5ex plus .2ex}%
                                {\normalfont\normalsize\bfseries}}
\renewcommand{\subsubsection}{\@startsection{subsubsection}{3}{0mm}%
                                {-1ex plus -.5ex minus -.2ex}%
                                {1ex plus .2ex}%
                                {\normalfont\small\bfseries}}
\makeatother

\setcounter{secnumdepth}{0}

\setlength{\parindent}{0pt}
\setlength{\parskip}{0pt plus 0.5ex}

\setitemize{noitemsep, topsep=0pt, parsep=0pt, partopsep=0pt}
\setenumerate{noitemsep, topsep=0pt, parsep=0pt, partopsep=0pt}

\newcommand{\Break}{\textbf{break}}
\newcommand{\Var}[1]{\textsf{#1}}
\newcommand{\alg}[1]{\textsc{\bfseries \footnotesize #1}}

\begin{document}
\raggedright
\begin{multicols}{3}
\setlength{\premulticols}{1pt}
\setlength{\postmulticols}{1pt}
\setlength{\multicolsep}{1pt}
\setlength{\columnsep}{2pt}

\begin{center}
     \Large{\underline{Algorithms Final Cheat Sheet}} \\
\end{center}

\section{Randomized Algorithms}

\subsection{Processes and DB}
Assume each process $P_i$ has some probability $p$ to access a database. A successful round is when exactly one process accesses the database. The probability of a single successful round is $p(1 - p)^{n-1}$.\\
We can maximize this probability by setting $p = \frac{1}{n}$. The probability of success here is $\Theta(\frac{1}{n})$.\\
Now, let's examine how long it will take process $P_i$ to succeed in accessing the database at least once. Running $t$ times, the probability that process $P_i$ does not succeed in any of rounds 1 through $t$ is $\leq (1 - \frac{1}{en})^t$. Setting $t$ to $\left\lceil en \right\rceil$ means the probability is $\leq e^{-1}$, independent of $n$.\\
If we set $t = \left\lceil en \right\rceil \cdot (c \ln n)$, the probability of failure is upper-bounded by $n^{-c}$.\\
With probablity at least $1 - n^{-1}$, all processes succeed in accessing the database at least once within $t = 2 \left\lceil en \right\rceil \ln n$ rounds.

\subsection{The Union Bound}
Prob [ $\bigcup\limits_{i=1}^{\infty} F_{i}$ ] <= $\sum_{n=1}^{\infty}  Prob[ F_{i}] $ 

%%%%%%%

\subsection{Verifying AB=C}

B$\overbar{r} -> A(B\overbar{r})$ and $C\overbar{r}$. if $A(B\overbar{r}) != C\overbar{r}$ then $AB != C$

\subsection{Principle of Deferred Decisions}
If $AB != C$ and $\overbar{r}$ is chosen uniformly at random with $r_n$ at 0-1 then Prob($AB\overbar{r} = C\overbar{r}) <= 1/2$

\subsection{Law of Total Probability}
Let $E_1 ... E_n$ be mutually disjoint events in the sample space $\Omega$ and let $\bigcup\limits_{i=1}^{n} E_{i}$ then $\sum_{i=1}^{n}  Prob[ B | E_{i}] Prob[E_{i}] $ 

Repeated trials increase the runtime to $\Theta(kn^2)$
If it returns false, then it this is right, but if it returns true, then it returns so with some probability of mistake.

%%%%%%%

\subsection{Median Value}
$\sum_{j=1}^{\infty}  j * Prob[ X = j ]  = (n + 1) / 2$ while Prob $[X = j] = 1/n$

To take exactly $j$ steps: Prob $[X = j] = (1 - p)^{j - 1} p$
$1/p$ for the first success

\subsection{Median Value}
Linearity of Expectation: Given 2 random vars $X$ and $Y$ in the same probability space, $E [X + Y] = E[ X ] + E [ Y ] $

\subsection{Memoryless guessing}
Expected correct: 1, independent of $n$

\subsection{Memory guessing}
$H(n) = \Theta(log(n))$ [harmonic series]

\subsection{Coupon collection:}
$E[ X_{j}] = n / (n - j); n $= number of; $j$ = collected; $( n - j ) / n$ of getting a new one
$E[X] = nH(n) = \Theta(nlog(n))$

\subsection{Conditional Probability}
$E[ X | \alpha ] =  \sum_{j=0}^{\infty} j *$ Prob $[ X = j | \alpha]$

\subsection{Max 3-SAT}
There is a randomized algo with polynomial expected run time that is guaranteed to produce a truth assignment satisfying at least a $7/8$ fraction of all clauses. We would need $8k$ trials to get the satisfying assignment.

\begin{algorithmic}[1]
	\Function{Select}{$S, k$}
		\State {$a_i \gets random var in S$}
		\For{each element $a_j$ of $S$}
			\State{$S^{-}$.append($a_j$) if $a_j < a_i$}
			\State{$S^{+}$.append($a_j$) if $a_j > a_i$}
		\EndFor
		\If{$S^{-} = k - 1$}
			\State{return $a_i$}
		\ElsIf{$S^{-} \geq k$}
			\State{Select($S^{-}, k$)}
		\Else
			\State{Select($S^{+}, k-1- |S^{-}|$)}
		\EndIf
	\EndFunction
\end{algorithmic}

%%%%

\subsection{Quicksort ExpectedComparisons}
$2nln(n) + O(n), \Omega(nlogn)$

%%%

\subsection{Uniform}

Prob $[h(x) = i] = 1/m$ for all $x$ and all $i$

\subsection{Universal}
Prob $[h(x) = h(y)] = 1/m$ for all $x != y$
\subsection{Near-universal}
Prob $[h(x) = h(y)] \leq 2/m$ for all $x != y$; E[$chainLen$] $\leq 2*\alpha;$ Runtime: $\Theta(1 + \alpha)$
\subsection{k-uniform}
Prob $ [ \wedge{}_{j=1}^{k} h(x_j) = i_j) ] = 1 / m^{k}$ for all distinct $x_1...x_k$ and all $i_1...i_k$
\subsection{Load Factor}
$\alpha = n/m$
\subsection{Balanced Binary Tree Search}
O(1 + log($chainLen$)) with any hash 
O(1 + log($\alpha$)) for uniform
\subsection{Recursively Hash}
$O(log_{m}{n})$
\subsection{Expected Search w/ binary probe}
O(1)
\section{Max Flows and Min Cuts}
\includegraphics[width=\linewidth]{images/edgecollapse.png}
\subsection{Blind Guess}
\begin{algorithmic}[1]
	\Function{GuessMinCut}{$G$}
		\For{$i \gets n, 2$}
			\State pick a random edge $e$ in $G$
			\State $G \gets G/e$
		\EndFor
		\State \Return the only cut in $G$
	\EndFunction
\end{algorithmic}
$P(n) = \frac{2}{n(n-1)}$

\subsection{Repeated Guessing}
\begin{algorithmic}[1]
	\Function{KargerMinCut}{$G$}
		\State $mink \gets \infty$
		\For{$i \gets 1, N$}
			\State $X \gets \Call{GuessMinCut}{G}$
			\If{$\left|X\right| < mink$}
				\State $mink \gets \left|X\right|$
				\State $minX \gets X$
			\EndIf
		\EndFor
		\State \Return $minX$
	\EndFunction
\end{algorithmic}
Set $N = c \binom{n}{2} \ln n$ for some constant $c$. $P(n) \geq 1 - \frac{1}{n^c}$. \alg{KargerMinCut} computes the min cut of any $n$-node graph with high probability in $O(n^4 \log n)$ time.

\subsection{Not-So-Blindly Guessing}
\begin{algorithmic}[1]
	\Function{Contract}{$G, m$}
		\For{$i \gets n, m$}
			\State pick a random edge $e$ in $G$
			\State $G \gets G/e$
		\EndFor
	\EndFunction
	\Function{BetterGuess}{$G$}
		\If{$G$ has more than 8 vertices}
			\State $G_1 \gets \Call{Contract}{G, n/\sqrt{2} + 1}$
			\State $G_2 \gets \Call{Contract}{G, n/\sqrt{2} + 1}$
			\State $X_1 \gets \Call{BetterGuess}{G_1}$
			\State $X_2 \gets \Call{BetterGuess}{G_2}$
			\State \Return min($X_1, X_2$)
		\Else
			\State use brute force
		\EndIf
	\EndFunction
\end{algorithmic}
$P(n) \geq 1/\log n$. The running time is $O(n^2 \log n)$.

\subsection{Flows}
A \emph{flow} is a function $f$ that satisfies the \emph{conservation constraint} at every vertex $v$: the total flow \emph{into} $v$ is equal to the total flow \emph{out} of $v$.\\

A flow $f$ is \emph{feasible} if $f(e) \leq c(e)$ for each edge $e$. A flow \emph{saturates} edge $e$ if $f(e) = f(c)$, and \emph{avoids} edge $e$ if $f(e) = 0$.

\includegraphics[width=\linewidth]{images/flow.png}

\subsection{Cuts}
A \emph{cut} is a partition of the vertices into disjoint subsets $S$ and $T$ - meaning $S \cup T = V$ and $S \cap T = \emptyset$ - where $s \in S$ and $t \in T$.\\

If we have a capacity function $c$, the \emph{capacity} of a cut is the sum of the capacities of the edges that start in $S$ and end in $T$. The definition is asymmetric; edges that start in $T$ and end in $S$ are unimportant. The \emph{min-cut problem} is to compute a cut whose capacity is as large as possible.

\includegraphics[width=\linewidth]{images/mincut.png}

\begin{theorem}[Maxflow Mincut Theorem]
	In any flow network, the value of the maximum flow is \emph{equal} to the capacity of the minimum cut.
\end{theorem}

\subsection{Residual Capacity}
\[
	c_f(u \rightarrow v) =
	\begin{cases}
		c(u \rightarrow v) - f(u \rightarrow v) & \text{if } u \rightarrow v \in E\\
		f(v \rightarrow u) & \text{if } v \rightarrow u \in E\\
		0 & \text{otherwise}
	\end{cases}
\]

\includegraphics[width=\linewidth]{images/residualgraph.png}

\subsection{Augmenting Paths}
Suppose there is a path $s = v_0 \rightarrow v_1 \rightarrow \cdots \rightarrow v_r$ in the residual graph $G_f$. This is an \emph{augmenting path}. Let $F = \text{min}_i c_f(v_i \rightarrow v_{i+1})$ denote the maximum amount of flow that we can push through the augmenting path in $G_f$. We can augment the flow into a new flow function $f'$:

\[
	f'(u \rightarrow v) =
	\begin{cases}
		f(u \rightarrow v) + F & \text{if } u \rightarrow v \in s\\
		f(u \rightarrow v) - F & \text{if } v \rightarrow u \in s\\
		f(u \rightarrow v) & \text{otherwise}
	\end{cases}
\]

\subsection{Ford-Fulkerson}
Starting with the zero flow, repeatedly augment the flow along \emph{any} path from $s$ to $t$ in the residual graph, until there is no such path.

\subsection{Further Work}
The fastest known maximum flow algorithm, announced by James Orlin in 2012, runs in $O(VE)$ time.
\section{Flow/Cut Applications}
\subsection{Edge-Disjoint Paths}
A set of paths in $G$ is \emph{edge-disjoint} if each edge in $G$ appears in at most one of the paths; several edge-disjoint paths may pass through the same vertex, however.\\

Assign each edge capacity 1. The number of edge-disjoint paths is exactly equal to the value of the flow. Using Orlin's algorithm is overkill; the the maximum flow has value at most $V - 1$, so Ford-Fulkerson's original augmenting path algorithm also runs in $O(\left|f^*\right| E) = O(VE)$ time.

\subsection{Vertex Capacities and Vertex-Disjoint Paths}
If we require the total flow into (and out of) any vertex $v$ other than $s$ and $t$ is at most some value $c(v)$, we transform the input into a new graph. We replace each vertex $v$ with two vertices $v_{in}$ and $v_{out}$, connected by an edge $v_{in} \rightarrow v_{out}$ with capacity $c(v)$, and then replace every directed edge $u \rightarrow v$ with the edge $u_{out} \rightarrow v_{in}$ (keeping the same capacity).\\

Computing the maximum number of \emph{vertex-disjoint} paths from $s$ to $t$ in any directed graph simply involves giving every vertex capacity 1, and computing a maximum flow.
\section{SAT and CNF-SAT}

\subsection{Formula Satisfiability/SAT}
$(a \vee b \vee c \vee \overbar{d})$ <=>  $((b \wedge \overbar{c}) \vee \overbar{ ( \overbar{a} => d) } \vee (c \neq a \wedge b)) $

\subsection{CNF}
conjunction/AND of several clauses which use OR inside these clauses.
\subsection{3CNF}
cnf with exactly 3 literals per clause

\subsection{Maximum Independent Set (from 3Sat)}
input is a simple, unweighted graph, get the size of the largest/smallest subgraph
Make the formula into a graph or vice versa, if it has an independent set of size $k$, its possible
Any graph has an edge-complement with the same vertices but the opposite set of edges if its not an edge in $G$. Its independent in $G$ if and only if the same vertices define a clique in $\overbar{G}$ (a complete graph). The largest independent is thus the largest clique in the compliment of the graph.

\includegraphics[width=\linewidth]{images/maxclique.png}

\subsection{3Color [From 3SAT]}
\includegraphics[width=\linewidth]{images/hamcycle.png}
\includegraphics[width=\linewidth]{images/truthgadget.png}
\includegraphics[width=\linewidth]{images/3color.png}

Truth gadget: $T,F$, and $X$ for true/false/other, variable gadget for variable a connecting and $\overbar{a}$ which must be opposite bools. Clause gadget joining three literal nodes to node $T$ in the truth gadget using give new unlabeled nodes and ten edges.
\section{NP-Hardness}
\begin{definition}
\emph{P} is the set of decision problems that can be solved in polynomial time. Intuitively, P is the set of problems that can be solved quickly.
\end{definition}
\begin{definition}
\emph{NP} is the set of decision problems with the following property: if the answer is \Var{Yes}, then there is a \emph{proof} of this fact that can be checked in polynomial time. Intuitively, NP is the set of decision problems where we can verify a \Var{Yes} answer quickly if we have the solution in front of us.
\end{definition}
\begin{definition}
\emph{co-NP} is essentially the opposite of NP. If the answer to a problem in co-NP is \Var{No}, then there is a proof of this fact that can be checked in polynomial time.
\end{definition}
Every decision problem in P is also in NP and also in co-NP.
\begin{definition}
A problem $\Pi$ is \emph{NP-hard} if a polynomial-time algorithm for $\Pi$ would imply a polynomial-time algorithm for every problem in NP.
\end{definition}
\begin{definition}
A problem $\Pi$ is \emph{NP-complete} if it is both NP-hard and an element of NP.
\end{definition}
\begin{theorem}[Cook-Levin Theorem]
Circuit satisfiability is NP-complete.
\end{theorem}
To prove that problem $A$ is NP-hard, reduce a known NP-hard problem to $A$.
\begin{definition}
A \emph{many-one} reduction from one language $L' \subseteq \Sigma^*$ is a function $f : \Sigma^* \rightarrow \Sigma^*$ such that $x \in L'$ iff $f(x) \in L$. A \emph{language} $L$ is NP-hard iff, for any language $L' \in NP$, there is a \emph{many-one} reduction from $L'$ to $L$ that can be computed in polynomial time.
\end{definition}
\subsection{NP-Hard Problems}
\begin{itemize}
	\item SAT
	\item 3SAT
	\item Maximum Independent Set: find the size of the largest subset of the vertices of a graph with no edges between them
	\item Clique: Compute the number of nodes in its largest complete subgraph
	\item Vertex Cover: Smallest set of vertices that touch every edge in the graph
	\item Graph Coloring: Find the smallest possible number of colors in a legal coloring such that every edge has two different colors at its endpoints
	\item Hamiltonian Cycle: find a cycle that visits each vertex in a graph exactly once
	\item Subset Sum: Given a set $X$ of positive integers and an integer $t$, determine whether $X$ has a subset whose elements sum to $t$
	\item Planar Circuit SAT: Given a boolean circuit that can be embedded in the plane so that no two wires cross, is there an input that makes the circuit output \Var{True}
	\item Not All Equal 3SAT: Given a 3CNF formula, is there an assignment of values to the variables so that every clause contains at least one \Var{True} literal \emph{and} at least one \Var{False} literal?
	\item Exact 3-Dimensional Matching: Given a set $S$ and a collection of three-element subsets of $S$, called \emph{triples}, is there a sub-collection of disjoint triples that exactly cover $S$?
	\item Partition: Given a set $S$ of $n$ integers, are there subsets $A$ and $B$ such that $A \cup B = S$, $A \cap B = \emptyset$, and $\sum_{a \in A} a = \sum_{b \in B} b$?
	\item 3Partition: Given a set $S$ of $3n$ integers, can it be partitioned into $n$ disjoint three-element subsets, such that every subset has exactly the same sum?
	\item Set Cover: Given a collection of sets $\mathscr{S} = \{ S_1, S_2, \ldots, S_m \}$, find the smallest sub-collection of $S_i$'s that contains all the elements of $\bigcup_i S_i$
	\item Hitting Set: Given a collection of sets $\mathscr{S} = \{ S_1, S_2, \ldots, S_m \}$, find the minimum number of elements of $\bigcup_i S_i$ that hit every set in $\mathscr{S}$
	\item Hamiltonian Path: Given a graph $G$, is there a path in $G$ that visits every vertex exactly once?
	\item Longest Path: Given a non-negatively weighted graph $G$ and two vertices $u$ and $v$, what is the longest simple path from $u$ to $v$ in the graph? A path is \emph{simple} if it visits each vertex at most once.
	\item Steiner Tree: Given a weighted, undirected graph $G$ with some of the vertices marked, what is the minimum-weight subtree of $G$ that contains every marked vertex?
\end{itemize}

\subsection{Hamiltonian Cycle [From Vertex Cover]}

For each vertex $u$, all the edge gadgets are connected in $H$ into a single directed path, a vertex chain. $H$ has $d-1$ additional edges for each $i$. $H$ also contains $k$ cover vertices, $1 - k$, with a directed edge to the first vertex in each vertex chain and a directed edge from the last vertex in each vertex chain.\\

Start at cover vertex 1 and traverse vertex chain for $vu_{2}$, then visit cover vertex 2 and so on and so forth before returning to 1. If $v$ is a part of the vertex cover, follow the edge from $(u_{i}, v, in)$ to $(u_{i}, v, out)$, else, detour from  $(u_{i}, v, in)$ > $(v, u_{i}, in)$ > $(v, u_{i}, out)$ > $(u_{i}, v, out)$.\\

$G$ contains a vertex cover of size $K$ if and only if $H$ contains a Hamiltonian cycle

\includegraphics[width=\linewidth]{images/edgegadget.png}
\includegraphics[width=\linewidth]{images/hamiltoniancycle.png}

\subsection{Subset Sum}

Given a graph $G$ and an integer $k$, first number edges from 0 to $m-1$; set $X$ contains the integer $b_{i} = 4^{i}$ for each edge $i$, and the integer $a_{v} = 4^{m} + \sum_{i \in \delta(v)}{4^i}$ where $\delta(v)$ is the set of edges that have $v$ as an endpoint. Finally, we set the target sum: $t = k * 4^m + \sum_{i = 0}^{m-1}{2 * 4^i}$

\subsection{Longest Increasing Subsequence [from DAG]}

Turn every number in the sequence into a vertex in a graph. Construct a special vertex $d$. For each vertex $v_1$, construct an edge to another vertex $v_2$ if (1) $v_2$ comes after $v_1$ in the sequence, and (2) $v_2$ > $v_1$. Also construct an edge to $d$.\\ 

Every path on this graph is a valid increasing subsequence. The problem of finding the LIS is now the problem of finding the longest path on this graph. Apply DAG.

\subsection{Maximum Independent Set [from 3SAT]}

Construct a graph $G$ which has one vertex for each instance of each literal in the 3SAT formula. Two vertices are connected by an edge if (1) they correspond to literals in the same clause, or (2) they correspond to a variable and its inverse. For example, the formula $(a \vee b \vee c) \wedge (b \vee \overline{c} \vee \overline{d}) \wedge (\overline{a} \vee c \vee d) \wedge (a \vee \overline{b} \vee \overline{d})$ is transformed into:

\includegraphics[width=\linewidth]{images/3CNFGraph.png}

Suppose the original formula had $k$ clauses. Then the formula is satisfiable iff the graph has an independent set of size $k$.

\subsection{Clique [from Independent Set]}
Any graph $G$ has an \emph{edge-complement} $\overline{G}$ with the same vertices, but with exactly the opposite set of edges - $(u, v)$ is an edge in $\overline{G}$ if and only if it is \emph{not} an edge in $G$. A set of vertices is independent in $G$ if and only if the same vertices define a clique in $\overline{G}$. Thus, we can compute the largest independent set in a graph by computing the largest clique in the complement of the graph.

\includegraphics[width=\linewidth]{images/maxclique.png}

\subsection{3Color [from 3SAT]}
\includegraphics[width=\linewidth]{images/truthgadget.png}
\includegraphics[width=\linewidth]{images/3color.png}

Truth gadget: $T,F$, and $X$ for true/false/other, variable gadget for variable a connecting and $\overbar{a}$ which must be opposite bools. Clause gadget joining three literal nodes to node $T$ in the truth gadget using give new unlabeled nodes and ten edges.

\end{multicols}
\end{document}
